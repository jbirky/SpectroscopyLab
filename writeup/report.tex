\documentclass[preprint]{aastex62}

% \usepackage{minted}
\usepackage{amsmath}
\usepackage{listings}
\usepackage{courier}
\usepackage{cleveref}
\usepackage{float}

\definecolor{bcolor}{RGB}{0, 51, 153}
\definecolor{gcolor}{RGB}{51, 153, 51}

\shorttitle{astronomical spectroscopy}
\shortauthors{j. birky}

\begin{document}

\title{\sc Lab 2: Astronomical Spectroscopy}
\author{Jessica Birky, Julian Beaz-Gonzalez, Russell Van-Linge}

\correspondingauthor{Jessica Birky (A13002163)}
\email{jbirky@ucsd.edu}

\begin{abstract}
In this lab... 

\end{abstract}
\bigskip

\section{Introduction} 
% Astronomical spectroscopy, components of spectrographs, types of spectrographs
Kirchoff's laws of spectral formation state that there are three different types of 

Section \ref{sec:observations} covers the schematics and specifications of two different spectrographs: the Ocean Lab spectrometer, and the KAST spectrograph mounted on the Shane 3m Telescope on Lick Observatory.


% ==================================
\section{Observations} \label{sec:observations}
Describe ocean lab and KAST spectrographs specifications and diagrams

% ==================================
\section{Data Reduction \& Methods} \label{sec:methods}
Centroid routine, wavelength calibration, centroid error and calibration error

Bias subtraction and normalization
\begin{equation}
\begin{bmatrix}
    m \\ c
\end{bmatrix}
= 
\begin{bmatrix}
    \sum x_i^2 & \sum x_i \\
    \sum x_i & N
\end{bmatrix}^{-1}
\begin{bmatrix}
    \sum x_i y_i \\ \sum y_i
\end{bmatrix}
\end{equation}

\begin{equation}
    \sigma^2 = \frac{1}{N-2}\sum_i [y_i - (mx_i + c)]^2
\end{equation}
\begin{equation}
    \sigma_m^2 = \frac{N\sigma^2}{N\sum_i x_i^2 - \left(\sum_i x_i \right)^2}
\end{equation}
\begin{equation}
    \sigma_c^2 = \frac{\sigma^2 \sum_i x_i^2}{N\sum_i x_i^2 - \left(\sum_i x_i \right)^2}
\end{equation}

% \begin{figure}[H]
% \plotone{plots/helium_reference.png}
% \caption{\href{https://www.vernier.com/innovate/a-quantitative-investigation-of-the-helium-spectrum/}{https://www.vernier.com/innovate/a-quantitative-investigation-of-the-helium-spectrum/}} \label{fig:bias}
% \end{figure}

% ==================================
\section{Data Analysis \& Modeling} \label{sec:analysis}
Show centroids, wavelength calibration plots, application of wavelength solution, plots of astronomical spectra

\begin{table}[H]
    \begin{center}
    \begin{tabular}{|c|c|c|c|c|}
    \hline
    spec\_pixel & spec\_intensity (ADU) & nist\_wave (nm) & nist\_intensity & element \\
    \hline \hline
    132.3649283 & 236.48          & 388        & 60-300          & He      \\
    288.9587617 & 249.46          & 447        & 25-200          & He      \\
    412.1908462 & 98.46           & 492        & 20              & He      \\
    437.7643945 & 271.16          & 501        & 100             & He      \\
    677.5521828 & 3106.25         & 588        & 120-500         & He      \\
    908.0143522 & 564.74          & 668        & 200             & He      \\
    1021.697778 & 867.6           & 706        & 20-100          & He      \\
    673.712     & 1696.45         & 585.25     & 200             & Ne      \\
    700.126     & 617.38          & 594.48     & 50              & Ne      \\
    722.492     & 181.31          & 602.99     & 100             & Ne      \\
    747.974     & 1339.68         & 614.3      & 100             & Ne      \\
    775.796     & 290.88          & 621.73     & 100             & Ne      \\
    820.02      & 3428.72         & 640.22     & 200             & Ne      \\
    860.822     & 1265.04         & 650.65     & 150             & Ne      \\
    886.09      & 419.28          & 659.9      & 100             & Ne      \\
    913.44      & 767.24          & 667.82     & 50              & Ne      \\
    982.83      & 635.18          & 692.94     & 1000            & Ne      \\
    1013.238    & 834.96          & 705.91     & 100             & Ne      \\
    1134.814    & 80.86           & 748.87     & 300             & Ne      \\
    \hline
    \end{tabular}
    \end{center}
\end{table}

\begin{table}[H]
    \begin{center}
    \begin{tabular}{|c|c|c|}
    \hline
    spec\_pixel & spec\_intensity & ref\_wave \\
    \hline \hline
    23.50092286 & 30772.15763     & 326.105   \\
    184.1577234 & 37829.79081     & 361.051   \\
    229.5492333 & 40120.66215     & 365.015   \\
    477.9985327 & 11444.21891     & 404.656   \\
    639.6619601 & 30172.47648     & 435.833   \\
    949.4331504 & 35609.38425     & 479.992   \\
    1257.08592  & 5622.686548     & 505.882   \\
    1372.254406 & 10136.48802     & 546.074   \\
    1638.903321 & 10132.41605     & 587.562   \\
    \hline
    \end{tabular}
    \end{center}
\end{table}

% ===================
\begin{figure}[]
\begin{center}
\includegraphics[width=.49\linewidth]{plots/oceanlab_helium_thres.png}
\includegraphics[width=.49\linewidth]{plots/oceanlab_neon_thres.png} \\
\includegraphics[width=.49\linewidth]{plots/oceanlab_helium_centroids.png}
\includegraphics[width=.49\linewidth]{plots/oceanlab_neon_centroids.png}
\caption{Centroid finding routine.}
\end{center}
\end{figure}

\begin{figure}[]
\begin{center}
\includegraphics[width=.49\linewidth]{plots/kast_arclamp_thres.png} \\
\includegraphics[width=.49\linewidth]{plots/kast_arclamp_centroids.png} 
\caption{Centroid finding routine.}
\end{center}
\end{figure}

% ===================
\begin{figure}[]
\begin{center}
\includegraphics[width=.46\linewidth]{plots/oceanlab_wavecal.png}
\includegraphics[width=.46\linewidth]{plots/oceanlab_residual.png}
\caption{Wavelength calibration for the Ocean Lab spectrometer.}
\end{center}
\end{figure}

\begin{figure}[]
\begin{center}
\includegraphics[width=.46\linewidth]{plots/kast_wavecal.png}
\includegraphics[width=.46\linewidth]{plots/kast_residual.png}
\caption{Wavelength calibration for the KAST spectrometer.}
\end{center}
\end{figure}

% ===================
\begin{figure}[]
\begin{center}
\includegraphics[width=.46\linewidth]{plots/oceanlab_error_vs_intensity2.png}
\includegraphics[width=.46\linewidth]{plots/oceanlab_error_vs_intensity.png}
\caption{Error vs. Intensity.}
\end{center}
\end{figure}

% ===================
% Science images
\begin{figure}[]
\begin{center}
\includegraphics[width=.31\linewidth]{plots/oceanlab_solar.png}
\includegraphics[width=.31\linewidth]{plots/oceanlab_fluorescent.png}
\includegraphics[width=.31\linewidth]{plots/oceanlab_lamp.png}
\caption{Three spectra taken on the Ocean Lab Spectrometer: the Sun, a fluorescent light bulb, and an incandescent lamp. Wavelength scale determined by calibration with emission peaks in Helium and Neon lamps.}
\end{center}
\end{figure}

\begin{figure}[]
\begin{center}
\includegraphics[width=.31\linewidth]{plots/kast_bd15233.png}
\includegraphics[width=.31\linewidth]{plots/kast_feige110.png}
\includegraphics[width=.31\linewidth]{plots/kast_j00470319.png}
\caption{Three astronomical spectra taken on the KAST spectrograph.}
\end{center}
\end{figure}

% ==================================
\section{Discussion} \label{sec:discussion}
What kind of astronomical sources

% ==================================
\section{Conclusion}


% ==================================
\section{Author Contributions}


% ==================================
\newpage
\section{Appendix}

\lstset{language=Python,
        basicstyle=\footnotesize\ttfamily,
        keywordstyle=\color{blue},
        numbers=left,
        numberstyle=\ttfamily,
        stringstyle=\color{red},
        commentstyle=\color{gcolor},
        morecomment=[l][\color{gray}]{\#}
}

\subsection{Centroid Identification Routine} \label{code:stats}
\small
\hrule
\begin{lstlisting}
def emission(data, **kwargs):
    """
    Input:  'data': 2D array [x,y]
            'thres': standard deviation threshold
    Output: 'emission': an array of arrays which contain the index of 
                each emission which lies above some threshold cut
    """
    thres = kwargs.get('thres', 1)
    
    x, y = np.array(data[0]), np.array(data[1])
    Npix = len(x)
    
    med = np.median(y)
    std = np.std(y)
    cut = med + thres*std
    
    count_cut = []
    for i in range(Npix):
        if y[i] >= cut:
            count_cut.append(i)
    
    emission = []
    arr = []
    for i in range(len(count_cut)-1):
        if (count_cut[i+1] - count_cut[i]) == 1:
            arr.append(count_cut[i])
        else:
            arr.append(count_cut[i])
            if len(arr) > 5:
                emission.append(np.array(arr))
            arr = []

return np.array(emission)

"""
Produces float values for the centroids, calculating their error and width
Credit: Julian
"""
centroids, errors, widths, intensities = [], [], [], []
for i in np.arange(len(cent)):
    inte = []
    for a in emiss[i]:
        inte.append(data[1][a])
    centr_f = sum(emiss[i]*inte)/sum(inte)
    err_f = sum(inte*((emiss[i]-centr_f)**2))/(sum(inte))**2
    width_f = sum(inte*((emiss[i]-centr_f)**2))/sum(inte)
    
    centroids.append(centr_f)
    errors.append(err_f)
    widths.append(width_f)
    intensities.append(max(data[1][emiss[i]]))
\end{lstlisting}
\hrule \vspace{7pt}


\subsection{Wavelength Calibration} \label{code:stats}
\small
\hrule
\begin{lstlisting}
def linear_regression(x, y):
    """
    Input:  x, y: 1D arrays
    Output: [m, c], [m_err, c_err]: slope and intercept best fit and error
    """
    N = len(x)
    x, y = np.array(x), np.array(y)

    A = np.array([[np.sum(x**2), np.sum(x)], \
                  [np.sum(x), N]])
    a = np.array([np.sum(x*y), np.sum(y)])

    fit = np.dot(np.linalg.inv(A), a)

    sig_sq = np.sum(y - (fit[0]*x + fit[1]))**2/(N + 2)
    m_err = np.sqrt(N*sig_sq/(N*np.sum(x**2) - (np.sum(x))**2))
    c_err = np.sqrt(sig_sq*np.sum(x**2)/(N*np.sum(x**2) - (np.sum(x))**2))
    err = np.array([m_err, c_err])

    return fit, err
\end{lstlisting}
\hrule \vspace{7pt}


\subsection{KAST Data Reduction} \label{code:stats}
\small
\hrule
\begin{lstlisting}
"""
Credit: Russell
"""
def readData(folder):
    '''
    Reads all the fits files in a folder and creates a 3D array
    '''
    files = os.listdir(folder)
    array3D = []
    for ff in files:
        arr = fits.getdata(folder+ff,ignore_missing_end=True)
        array3D.append(arr)
    return np.array(array3D)

def combineFrame(data_array):
    '''
    Avergaes the 3D into 2D array
    '''
    return np.median(data_array,axis=0)

def slit_size(data = 'fits_file'):
    '''
    takes a fits file and finds the slit sized used
    '''
    dt = fits.open(data)
    header = dt[0].header
    print(header['SLIT_N'])
    
    
def norm_flat(bias_folder,flat_folder):
    '''
    Input the bias and flat folders and creates
    bias,flats, and normalized flats
    '''
    flat3d = readData(flat_folder)
    bias3d = readData(bias_folder)

    flat2d = combineFrame(flat3d)
    bias2d = combineFrame(bias3d)

    norm_flat = (flat2d-bias2d)/np.median(flat2d-bias2d)
    return norm_flat,flat2d,bias2d

def science_image(science_folder,flat_folder,bias_folder):
    '''
    Creates a science image and outputs the 2D array to be used to make spectra
    '''
    # create flats and bias
    normflat, flat, bias = norm_flat(bias_folder,flat_folder) 

    science_3d = readData(science_folder) # reducing down to 2d 
    science_2d = combineFrame(science_3d)
    science_final = (science_2d - bias)/normflat # subtracting bias and normalizing
    # Plots the science image
    plt.imshow(science_final,origin='lower',interpolation='nearest',\
        cmap='gray',vmin=10,vmax=1000)
    plt.show()
    return science_final

def science_spectra(science_2d,start_slice,end_slice,file_name='file_name'):
    '''
    Creates a spectra by slicing up the science image at the input spatial
    pixels, saves array, and outputs plot
    '''
    spectra = np.mean(science_2d[start_slice:end_slice,:],axis=0)
    spectra = spectra[spectra>0]
    x = np.arange(0,len(spectra))
    plt.plot(x,spectra,color='black')
    plt.show()
    np.save('%s_spectra'%file_name,science_2d)
    return spectra
\end{lstlisting}
\hrule \vspace{7pt}



\end{document}

