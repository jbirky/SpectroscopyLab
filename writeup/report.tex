\documentclass[preprint]{aastex62}

% \usepackage{minted}
\usepackage{amsmath}
\usepackage{listings}
\usepackage{courier}
\usepackage{cleveref}
\usepackage{float}

\definecolor{bcolor}{RGB}{0, 51, 153}
\definecolor{gcolor}{RGB}{51, 153, 51}

\shorttitle{astronomical spectroscopy}
\shortauthors{j. birky}

\begin{document}

\title{\sc Lab 2: Astronomical Spectroscopy}
\author{Jessica Birky, Julian Beaz-Gonzalez, Russell Van-Linge}

\correspondingauthor{Jessica Birky (A13002163)}
\email{jbirky@ucsd.edu}

\begin{abstract}
In this lab... 

\end{abstract}
\bigskip

\section{Introduction} 
Astronomical spectroscopy, components of spectrographs, types of spectrographs

% ==================================
\section{Observations} \label{sec:observations}
Describe ocean lab and KAST spectrographs specifications and diagrams

% ==================================
\section{Data Reduction \& Methods} \label{sec:methods}
Centroid routine, wavelength calibration, centroid error and calibration error

Bias subtraction and normalization
% \begin{figure}[H]
% \plotone{plots/helium_reference.png}
% \caption{\href{https://www.vernier.com/innovate/a-quantitative-investigation-of-the-helium-spectrum/}{https://www.vernier.com/innovate/a-quantitative-investigation-of-the-helium-spectrum/}} \label{fig:bias}
% \end{figure}

% ==================================
\section{Data Analysis \& Modeling} \label{sec:analysis}
Show centroids, wavelength calibration plots, application of wavelength solution, plots of astronomical spectra

\begin{figure}[]
\begin{center}
\includegraphics[width=.49\linewidth]{plots/oceanlab_helium_thres.png}
\includegraphics[width=.49\linewidth]{plots/oceanlab_neon_thres.png} \\
\includegraphics[width=.49\linewidth]{plots/oceanlab_helium_centroids.png}
\includegraphics[width=.49\linewidth]{plots/oceanlab_neon_centroids.png}
\caption{Centroid finding routine.}
\end{center}
\end{figure}

\begin{figure}[]
\begin{center}
\includegraphics[width=.46\linewidth]{plots/oceanlab_wavecal.png}
\includegraphics[width=.46\linewidth]{plots/oceanlab_residual.png}
\caption{Wavelength calibration for the Ocean Lab spectrometer.}
\end{center}
\end{figure}


\begin{figure}[]
\begin{center}
\includegraphics[width=.31\linewidth]{plots/oceanlab_solar.png}
\includegraphics[width=.31\linewidth]{plots/oceanlab_fluorescent.png}
\includegraphics[width=.31\linewidth]{plots/oceanlab_lamp.png}
\caption{Three spectra taken on the Ocean Lab Spectrometer: the Sun, a fluorescent light bulb, and an incandescent lamp. Wavelength scale determined by calibration with emission peaks in Helium and Neon lamps.}
\end{center}
\end{figure}

% ==================================
\section{Discussion} \label{sec:discussion}
What kind of astronomical sources

% ==================================
\section{Conclusion}


% ==================================
\section{Author Contributions}


% ==================================
\newpage
\section{Appendix}

\lstset{language=Python,
        basicstyle=\footnotesize\ttfamily,
        keywordstyle=\color{blue},
        numbers=left,
        numberstyle=\ttfamily,
        stringstyle=\color{red},
        commentstyle=\color{gcolor},
        morecomment=[l][\color{gray}]{\#}
}

\subsection{Centroid Identification Routine} \label{code:stats}
\small
\hrule
\begin{lstlisting}
def emission(data, **kwargs):
    thres = kwargs.get('thres', 1)
    
    x, y = np.array(data[0]), np.array(data[1])
    Npix = len(x)
    
    med = np.median(y)
    std = np.std(y)
    cut = med + thres*std
    
    count_cut = []
    for i in range(Npix):
        if y[i] >= cut:
            count_cut.append(i)
    
    emission = []
    arr = []
    for i in range(len(count_cut)-1):
        if (count_cut[i+1] - count_cut[i]) == 1:
            arr.append(count_cut[i])
        else:
            arr.append(count_cut[i])
            if len(arr) > 5:
                emission.append(np.array(arr))
            arr = []

return np.array(emission)


def centroid(data, feat_idx, **kwargs):
    
    x, y = np.array(data[0]), np.array(data[1])
    Npix = len(x)
    
    max_idx = []
    for feat in feat_idx:
        for idx in feat:
            if y[idx] == max(y[feat]):
                max_idx.append(idx)
                
    return np.array(max_idx)

centroid(data, emiss)
\end{lstlisting}
\hrule \vspace{7pt}

\subsection{Centroid Error} \label{code:stats}
\small
\hrule
\begin{lstlisting}
#Produces float values for the centroids, calculating their error and width
centroids, errors, widths, intensities = [], [], [], []
for i in np.arange(len(cent)):
    inte = []
    for a in emiss[i]:
        inte.append(data[1][a])
    centr_f = sum(emiss[i]*inte)/sum(inte)
    err_f = sum(inte*((emiss[i]-centr_f)**2))/(sum(inte))**2
    width_f = sum(inte*((emiss[i]-centr_f)**2))/sum(inte)
    
    centroids.append(centr_f)
    errors.append(err_f)
    widths.append(width_f)
    intensities.append(max(data[1][emiss[i]]))
\end{lstlisting}
\hrule \vspace{7pt}

\subsection{Wavelength Calibration} \label{code:stats}
\small
\hrule
\begin{lstlisting}
def linear_regression(x, y):
    
    A = np.array([[np.sum(x**2), np.sum(x)], \
                  [np.sum(x), len(x)]])
    a = np.array([np.sum(x*y), np.sum(y)])

    return np.dot(np.linalg.inv(A), a)
\end{lstlisting}
\hrule \vspace{7pt}



\end{document}

